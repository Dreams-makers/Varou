\section{Description}
Les glyphs sont des mouvements à effectuer à la souris afin de lancer un sort. Ces glyphs seront reconnu par un systeme de reconnaissance adaptatifs.
Si le mouvement est trop loin des glyphs connus ou possible, aucun sort ne sera lancé.
\section{Reconnaissance}
L'entité glyph est exactement un tracé d'une ligne courbe. Le taux de correlation avec la liste prédéfinie (un glyph par sort) permetera une différentiation et un taux réussite du sort plus ou moins élevé.
Une première idée est d'utiliser une reconnaissance par correlation d'un chemin vectoriel. Un glyph n'est pas comme une lettre curviligne où un traitement d'image serait necessaire, car il y plusieurs traits. Un glyph possède un point de départ et un point d'arret. En discrétisant le chemin parcouru en une série vectoriel, le gradient de celle-ci pourrait offrir un moyen de comparaison avec une liste de modèle.
Idéalement, le calcul serait à répartir en une liste d'équations résolvables en une étape, par inversion de matrice.